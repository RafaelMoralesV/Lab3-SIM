% Template:     Informe LaTeX
% Documento:    Archivo principal
% Versión:      8.1.6 (12/06/2022)
% Codificación: UTF-8
%
% Autor: Pablo Pizarro R.
%        pablo@ppizarror.com
%
% Manual template: [https://latex.ppizarror.com/informe]
% Licencia MIT:    [https://opensource.org/licenses/MIT]

% CREACIÓN DEL DOCUMENTO
\documentclass[
	spanish, % Idioma: spanish, english, etc.
	letterpaper, oneside
]{article}

% INFORMACIÓN DEL DOCUMENTO
\def\documenttitle {Laboratorio 3}
\def\documentsubtitle {}
\def\documentsubject {Sistema de cajas en un banco}

\def\documentauthor {Rafael Morales Venegas}
\def\coursename {Simulación de Sistemas}
\def\coursecode {INFB8093}

\def\universityname {Universidad Tecnológica Metropolitana}
\def\universityfaculty {Facultad de Ingeniería, Campus Macúl}
\def\universitydepartment {Escuela de Informática}
\def\universitydepartmentimage {utem_logo}
\def\universitydepartmentimagecfg {height=1.57cm}
\def\universitylocation {Santiago de Chile}

% INTEGRANTES, PROFESORES Y FECHAS
\def\authortable {
\begin{tabular}{ll}
	Integrantes:
	& \begin{tabular}[t]{l}
		Rafael Morales Venegas
	\end{tabular} \\
	Profesor:
	& \begin{tabular}[t]{l}
		Santiago Enrique Zapata Cáceres
	\end{tabular} \\
	% Auxiliar:
	% & \begin{tabular}[t]{l}
	% 	Auxiliar 1
	% \end{tabular} \\
	% Ayudantes:
	% & \begin{tabular}[t]{l}
	% 	Ayudante 1 \\
	% 	Ayudante 2
	% \end{tabular} \\
	% \multicolumn{2}{l}{Ayudante de laboratorio: Ayudante 1} \\
	% & \\
	\multicolumn{2}{l}{Fecha de realización: \today} \\
	\multicolumn{2}{l}{Fecha de entrega: 07 de julio de 2022} \\
	\multicolumn{2}{l}{\universitylocation}
\end{tabular}
}

% IMPORTACIÓN DEL TEMPLATE
\input{template}
\usepackage[utf8]{inputenc}
\usepackage{pgfplots}
\pgfplotsset{compat=1.18, width=\textwidth}

% INICIO DE PÁGINAS
\begin{document}

% PORTADA
\templatePortrait

% CONFIGURACIÓN DE PÁGINA Y ENCABEZADOS
\templatePagecfg

% RESUMEN O ABSTRACT
% \begin{abstractd}
% 	\lipsum[1] % Párrafo ejemplo, se puede borrar
% \end{abstractd}

% TABLA DE CONTENIDOS - ÍNDICE
\templateIndex

% CONFIGURACIONES FINALES
\templateFinalcfg

% ======================= INICIO DEL DOCUMENTO =======================

% \input{example.tex}
\section{Introducción}
Se presenta a continuación un problema de servicio y colas de un banco. El desarrollo del mismo será realizado mediante el software WinQSB, que permite una simulación y análisis del mismo.

\subsection{Problema}
Un banco tiene dos cajeros automáticos, que siguen una distribución exponencial, atienden a razón de 1,5 clientes/minuto, la tasa de llegadas de clientes sigue una distribución de Poisson y es igual a 30 clientes/hora. Se pide:

\begin{enumerate}[label=\Alph*]
    \item Número promedio de clientes en el sistema
    \item Tiempo promedio de un cliente en el sistema
    \item Porcentaje de tiempo de cajero libre
    \item Sensibilidad del sistema en 24 horas, usando el comando “simulate the system”, “semilla random por defecto”, “disciplina FIFO”, interprete cada una de las “performance measure” o “medidas de rendimiento”
    \item Análisis de sensibilidad para el parámetro “tasa de llegada λ = 30 clientes/hora”, con una variación de 30 a 100 clientes/horas, con un incremento de 10 clientes/hora, use el comando “solve and analyze”, “arrival rate λ”, “approximation by G/G/s” con start = 30, end = 100, step = 10; Gráfico de sensibilidad, use comando “Results”, “show sensitive analysis - graph”
    \item Haga un Análisis de Sensibilidad desde 2 servidores hasta 8 servidores con un paso de 1, con la siguiente información complementaria: costo servidor ocupado/hora = 5, costo servidor ocioso/hora = 1, costo cliente en espera = 0,5, costo cliente servido/hora = 3, costo cliente no atendido = 1, costo unitario capacidad de la cola = 3; use comando “perform capacity analysis”, “approximation by G/G/s”, N° server = 2, start = 2, end = 8, step = 1
\end{enumerate}

\pagebreak
\section{Desarrollo}
\subsection{Datos provistos}
Se entrega a continuación una traducción de los datos que se proveyeron en el problema:

Sea $\mu$ la Tasa de atención, tal que:
\insertaligned[\label{eqn:tasa-serv}]{
    \mu &= 1.5 \text{ clientes / minutos} \\
    \mu &= 1.5 \frac{c}{m} \cdot \frac{60 \cdot m}{1 \cdot h} \\
    \mu &= 90 \frac{c}{h} \text{ (Clientes por hora) }
}

Sea $\lambda$ la Tasa de llegada de clientes, tal que:
\insertaligned[\label{eqn:tasa-llegada}]{
    \lambda &= 30 \text{ Clientes por hora } \\
    \lambda &= 30 \frac{c}{h}
}

Además, tenemos los siguientes datos:
\begin{itemize}
    \item Dos cajeros, representando servidores.
    \item Cola, punto de reunión de los clientes, con atención FIFO.
    \item Clientes, como representación de todas las fuentes de aparición de clientes.
\end{itemize}
\pagebreak

\subsection{Simulación}

\insertimage[\label{img:winqsb1}]{winqsb/1.png}{scale=1}{Imagen de WinQSB}
\insertimage[\label{img:winqsb2}]{winqsb/2.png}{scale=1}{Imagen de WinQSB}
\insertimage[\label{img:winqsb3}]{winqsb/3.png}{width=\textwidth}{Imagen de WinQSB}
\insertimage[\label{img:winqsb4}]{winqsb/4.png}{scale=1}{Imagen de WinQSB}
\insertimage[\label{img:winqsb5}]{winqsb/5.png}{width=\textwidth}{Imagen de WinQSB}
\insertimage[\label{img:winqsb6}]{winqsb/6.png}{width=\textwidth}{Imagen de WinQSB}
\insertimage[\label{img:winqsb7}]{winqsb/7.png}{width=\textwidth}{Imagen de WinQSB}
\insertimage[\label{img:winqsb8}]{winqsb/8.png}{width=\textwidth}{Imagen de WinQSB}
\insertimage[\label{img:winqsb9}]{winqsb/9.png}{scale=1}{Imagen de WinQSB}
\insertimage[\label{img:winqsb10}]{winqsb/10.png}{width=\textwidth}{Imagen de WinQSB}
\insertimage[\label{img:winqsb11}]{winqsb/11.png}{width=\textwidth}{Imagen de WinQSB}
\insertimage[\label{img:winqsb12}]{winqsb/12.png}{width=\textwidth}{Imagen de WinQSB}
\insertimage[\label{img:winqsb13}]{winqsb/13.png}{width=\textwidth}{Imagen de WinQSB}
\insertimage[\label{img:winqsb14}]{winqsb/14.png}{width=\textwidth}{Imagen de WinQSB}


\pagebreak
\section{Conclusión}
El uso de WinQSB como herramienta para la simulación de problemas de cola significó la habilitación de simulación real con probabilidades y azar menos controlables (o más impredecibles) que las que podría generar una persona natural, por lo que lo vuelve útil para estudiar modelos de sistemas de colas de forma ligeramente más real.

Mi crítica, sin embargo, se ha mantenido constante contra este software. Es abandonware\footnote{Software que ya no es mantenido, o que ha sido abandonado por su desarrollador o desarrolladores.}, y la forma de utilización mediante el uso de máquinas virtuales para sistemas operativos que ya están fuera de su apoyo por parte de su desarrollador, hace que sea difícil su uso para problemas reales, sumado con la pobre interfaz del mismo.

Dicho esto, los resultados que entrega son lo suficientemente claros como para poder ser comprensibles sin problema, y el amplio repertorio de herramientas lo vuelve útil para análisis más complejos, como es el que se hizo con ambos análisis de sensibilidad.

Tomando el punto de estos análisis de sensibilidad, es importante notar como la aparición de nuevos servidores no fue útil para la empresa en cuanto a tiempos de atención, pues comparando y tomando en cuenta la sensibilidad de la tasa de llegada, es apreciable como el sistema está más que equipado para atender tanto la clientela actual, como una hasta el doble de intensa.


% FIN DEL DOCUMENTO
\end{document}