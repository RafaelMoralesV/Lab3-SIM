\section{Desarrollo}
\subsection{Datos provistos}
Se entrega a continuación una traducción de los datos que se proveyeron en el problema:

Sea $\mu$ la Tasa de atención, tal que:
\insertaligned[\label{eqn:tasa-serv}]{
    \mu &= 1.5 \text{ clientes / minutos} \\
    \mu &= 1.5 \frac{c}{m} \cdot \frac{60 \cdot m}{1 \cdot h} \\
    \mu &= 90 \frac{c}{h} \text{ (Clientes por hora) }
}

Sea $\lambda$ la Tasa de llegada de clientes, tal que:
\insertaligned[\label{eqn:tasa-llegada}]{
    \lambda &= 30 \text{ Clientes por hora } \\
    \lambda &= 30 \frac{c}{h}
}

Además, tenemos los siguientes datos:
\begin{itemize}
    \item Dos cajeros, representando servidores.
    \item Cola, punto de reunión de los clientes, con atención FIFO.
    \item Clientes, como representación de todas las fuentes de aparición de clientes.
\end{itemize}
\pagebreak

\subsection{Simulación}
\subsubsection{Nuevo problema}
Para resolver este problema, nos dirigiremos a la aplicación de análisis de cola (\quotes{Queuing Analysis}), donde se nos presentará la siguiente pantalla inicial, al intentar generar un nuevo proyecto:
\insertimage[\label{img:winqsb1}]{winqsb/1.png}{scale=0.7}{Generación de un problema en la aplicación \quotes{Queuing Analysis}}

Dar \quotes{OK} nos mostrará la siguiente tabla, que rellenaremos con los valores entregados anteriormente, nuestra tasa de atención (Ecuación \refeq{eqn:tasa-serv}), y nuestra tasa de llegada (Ecuación \refeq{eqn:tasa-llegada}), tal como se muestra en la siguiente figura:
\insertimage[\label{img:winqsb2}]{winqsb/2.png}{scale=0.7}{Tabla con los datos iniciales provistos por el problema.}

Ingresar el comando de resolver el problema nos arroja la siguiente tabla de análisis de desempeño:
\insertimage[\label{img:winqsb3}]{winqsb/3.png}{width=\textwidth}{Tabla de medición de desempeño}

Esta tabla contiene datos interesantes. Revisándola, podemos obtener la siguiente información:
\begin{itemize}
    \item La cantidad promedio de clientes en el sistema (Correspondiente a $L$, o $L_w$) es de tan solo $0.3429$.
    \item El porcentaje de utilización del sistema es tan solo de un $16.6667\%$, lo que habla de un tiempo de ocio de al rededor de un $83.333\%$.
    \item El tiempo promedio de un cliente en el sistema es de 0.0114 horas, o al rededor de 41 segundos!
    \item El tiempo de espera promedio, de forma similar, es también extremadamente corto, con un valor de $0.0003$ horas, o al rededor de 1 segundo.
\end{itemize}

\subsubsection{Simulación del sistema}
A continuación, entonces, se utilizará el comando \quotes{Solve and Analyze} , en su opción de \quotes{Simulate the System}, para generar una simulación de este sistema. Se presenta la pantalla emergente para la especificación de esta simulación, que se hará de acuerdo a los datos mostrados en la siguiente figura:
\insertimage[\label{img:winqsb4}]{winqsb/4.png}{scale=0.6}{Especificación de la simulación}

Similar a la figura \ref{img:winqsb3}, obtenemos una tabla de medida de desempeño a partir de una simulación:
\insertimage[\label{img:winqsb5}]{winqsb/5.png}{width=\textwidth}{Imagen de WinQSB}

Haciendo un análisis similar y comparativo entre ambas tablas, podemos ver lo siguiente:

\begin{itemize}
    \item Por supuesto, la tasa de llegada y servicio es idéntica.
    \item La tasa efectiva de llegada y atención, sin embargo, se muestran como $27.8927$ y $27.8511$, respectivamente.
    \item La utilización del sistema baja ligeramente a un $15.6925\%$.
    \item El número promedio de clientes en el sistema baja, también ligeramente, a $0.3225$.
    \item El tiempo de atención y espera en la cola son efectivamente idénticos, variando el primero únicamente en menos de 1 segundo.
    \item La probabilidad de que el sistema esté en ocio, o de que un cliente espere, se mantienen efectivamente iguales.
\end{itemize}

\subsubsection{Análisis de sensibilidad}
Generaremos un nuevo análisis de sensibilidad desde el comando de resultados, con la variación de nuestra tasa de llegada. Este valor variará desde los 30 a los 100 clientes por hora, en saltos de 10. La especificación para esto se muestra a continuación:

\insertimage[\label{img:winqsb6}]{winqsb/6.png}{scale=0.7}{Especificación para análisis de sensibilidad para la tasa de llegada}

La siguiente tabla (a la cual se le ha omitido la mitad de las columnas, debido a su poca relevancia y en ayuda del formato del documento) muestra los resultados de este análisis de sensibilidad:

\insertimage[\label{img:winqsb7}]{winqsb/7.png}{width=\textwidth}{Tabla de resultados del análisis de sensibilidad para la tasa de llegada}

Incluido además, se encuentra el gráfico de personas promedio en el sistema para este análisis de sensibilidad. Podemos ver como el valor ronda entre los 0.3 a 1.6 clientes en promedio dentro del sistema:

\insertimage[\label{img:winqsb8}]{winqsb/8.png}{width=\textwidth}{Gráfico de Clientes promedio en el sistema versus tasa de llegada.}

\subsubsection{Análisis de sensibilidad para servidores}
Se hará, entonces, un nuevo análisis de sensibilidad, esta vez para los servidores. Para este nuevo análisis, es relevante cambiar el problema original presentado en la figura \ref{img:winqsb2}, para incluir el variado costo propuesto en la opción G entregado en la especificación original del problema.

\insertimage[\label{img:winqsb9}]{winqsb/9.png}{scale=0.7}{Alteración a la tabla de datos iniciales para incluir los variados costos del punto G del problema original}

Se hará, entonces, un nuevo análisis de sensibilidad centrado en el número de los servidores. Este análisis se hará en pasos de 1, iniciando en 2 servidores, hasta los 8.

\insertimage[\label{img:winqsb10}]{winqsb/10.png}{scale=0.6}{Parámetros de análisis de sensibilidad para servidores.}

Se entrega, entonces, la tabla de resultados para el análisis de sensibilidad, con la mitad anteriormente omitida en la figura \ref{img:winqsb7}, que contiene la información de costos para este análisis.
\begin{images}[\label{img:winqsb11-12}]{Tabla de resultados del análisis de sensibilidad para el número de servidores.}
    \addimage[\label{img:winqsb11}]{winqsb/11.png}{width=\textwidth}{Tabla de resultados, parte 1}
    \addimage[\label{img:winqsb12}]{winqsb/12.png}{width=\textwidth}{Tabla de resultados, parte 2}
\end{images}

Las siguientes figuras, además, representan los gráficos para el costo total en función del número de servidores, y la utilización del sistema en la misma función.

\insertimage[\label{img:winqsb13}]{winqsb/13.png}{width=\textwidth}{Costo total versus número de servidores}
\insertimage[\label{img:winqsb14}]{winqsb/14.png}{width=\textwidth}{Utilización del sistema versus número de servidores.}


\pagebreak