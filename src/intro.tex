\section{Introducción}
Se presenta a continuación un problema de servicio y colas de un banco. El desarrollo del mismo será realizado mediante el software WinQSB, que permite una simulación y análisis del mismo.

\subsection{Problema}
Un banco tiene dos cajeros automáticos, que siguen una distribución exponencial, atienden a razón de 1,5 clientes/minuto, la tasa de llegadas de clientes sigue una distribución de Poisson y es igual a 30 clientes/hora. Se pide:

\begin{enumerate}[label=\Alph*]
    \item Número promedio de clientes en el sistema
    \item Tiempo promedio de un cliente en el sistema
    \item Porcentaje de tiempo de cajero libre
    \item Sensibilidad del sistema en 24 horas, usando el comando “simulate the system”, “semilla random por defecto”, “disciplina FIFO”, interprete cada una de las “performance measure” o “medidas de rendimiento”
    \item Análisis de sensibilidad para el parámetro “tasa de llegada λ = 30 clientes/hora”, con una variación de 30 a 100 clientes/horas, con un incremento de 10 clientes/hora, use el comando “solve and analyze”, “arrival rate λ”, “approximation by G/G/s” con start = 30, end = 100, step = 10; Gráfico de sensibilidad, use comando “Results”, “show sensitive analysis - graph”
    \item Haga un Análisis de Sensibilidad desde 2 servidores hasta 8 servidores con un paso de 1, con la siguiente información complementaria: costo servidor ocupado/hora = 5, costo servidor ocioso/hora = 1, costo cliente en espera = 0,5, costo cliente servido/hora = 3, costo cliente no atendido = 1, costo unitario capacidad de la cola = 3; use comando “perform capacity analysis”, “approximation by G/G/s”, N° server = 2, start = 2, end = 8, step = 1
\end{enumerate}

\pagebreak