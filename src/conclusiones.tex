\section{Conclusión}
El uso de WinQSB como herramienta para la simulación de problemas de cola significó la habilitación de simulación real con probabilidades y azar menos controlables (o más impredecibles) que las que podría generar una persona natural, por lo que lo vuelve útil para estudiar modelos de sistemas de colas de forma ligeramente más real.

Mi crítica, sin embargo, se ha mantenido constante contra este software. Es abandonware\footnote{Software que ya no es mantenido, o que ha sido abandonado por su desarrollador o desarrolladores.}, y la forma de utilización mediante el uso de máquinas virtuales para sistemas operativos que ya están fuera de su apoyo por parte de su desarrollador, hace que sea difícil su uso para problemas reales, sumado con la pobre interfaz del mismo.

Dicho esto, los resultados que entrega son lo suficientemente claros como para poder ser comprensibles sin problema, y el amplio repertorio de herramientas lo vuelve útil para análisis más complejos, como es el que se hizo con ambos análisis de sensibilidad.

Tomando el punto de estos análisis de sensibilidad, es importante notar como la aparición de nuevos servidores no fue útil para la empresa en cuanto a tiempos de atención, pues comparando y tomando en cuenta la sensibilidad de la tasa de llegada, es apreciable como el sistema está más que equipado para atender tanto la clientela actual, como una hasta el doble de intensa.
